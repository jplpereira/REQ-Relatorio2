\chapter{Contexto do Negócio}

  \section{Resumo}

  A empresa escolhida foi o Centro de Ensino Fundamental 03 do Gama (CEF 03 do
  Gama), escola da rede pública de educação do Distrito Federal, localizada na
  região administrativa do Gama e que tem a necessidade de possuir seu próprio
  sistema capaz de realizar o controle de assiduidade dos alunos e permitir
  somente a entrada dos regularmente matriculados na instituição. Para esses
  fins, a escola requisitou um software que seja capaz de realizar as tarefas
  supracitadas.

  \section{Histórico}

  Anteriormente, o controle de assiduidade dos alunos somente era realizado
  através dos diários de classe, que eram de papel e preenchidos à mão pelos
  professores. Eles eram recolhidos pela direção após o término de cada turno e
  armazenados na própria escola. Desta forma, o acesso às informações sobre a
  frequência dos alunos deveria ser realizado através de uma consulta manual
  nos diários da turma daquele aluno.

  A partir do ano de 2012, o Governo do Distrito Federal criou o DICEL (Diário
  de Classe Eletrônico) que era uma planilha de Excel com macros e cuja adesão
  por parte dos professores era opcional. Sendo assim, a direção da escola
  passou a ter informações sobre a assiduidade apenas questionando aos
  professores ou ao final de cada bimestre, quando os professores entregavam os
  diários fechados. Em alguns casos, a direção, quando questionada sobre a
  presença de algum aluno por um responsável, ao não conseguir entrar em contato
  com um professor, tinha que ligar para um colega de classe do aluno para saber
  se ele esteve presente.

  Por conta destes problemas, a escola se viu obrigada a fazer o controle de
  assiduidade dos alunos por conta própria e acabou por contratar uma empresa
  para implementar um sistema para realizá-lo.

  \section{O sistema atual}

  O controle de acesso é realizado através da identificação dos alunos
  matriculados por meio de uma carteira estudantil com uma cor específica de
  cada segmento. Contém em seu verso um código de barras que deve ser lido
  através de um leitor ótico instalado na portaria da escola. Desta forma,
  apenas os portadores da identificação têm acesso às dependências da escola.

  No ato da leitura do código de barras, o software registra a data e o horário
  de entrada de cada aluno para consulta imediata ou futuros relatórios sobre a
  sua frequência de presença. Esses relatórios são principalmente dos alunos de
  baixa renda que de famílias beneficiárias de bolsas governamentais cujo
  recebimento é condicionado ao compromisso dos filhos na escola.

  As carteiras estudantis são recolhidas após a leitura do código de barras,
  separadas pelas cores de cada segmento e pela turma e entregues apenas no
  horário da saída de cada uma das turmas.

  A empresa também oferece um serviço extra de envio de SMS para o celular do
  responsável do aluno, avisando sobre o horário de entrada e de saída da
  escola. Para isso, o responsável interessado deve pagar uma taxa semestral de
  R\$ 20,00.

  \section{A solução proposta}

  O sistema atual depende da contratação de uma empresa que mantém o sistema. A
  escola não recebe verba específica do governo para o seu pagamento, apenas uma
  parte do valor é custeada com o valor pago pelos alunos para a emissão das
  carteiras estudantis. Por conta da incerteza do recebimento de verba ou de um
  possível aumento no valor do serviço pago, a escola tem a necessidade de ter
  um sistema próprio para garantir a continuidade do controle de frequência dos
  alunos.

  A ideia deste projeto é desenvolver um sistema com as mesmas funcionalidades
  do existente, corrigir a necessidade de cadastrar os novos alunos em duas
  bases de dados diferentes e incluir novas funcionalidades como, por exemplo,
  uma forma de o próprio responsável poder consultar o horário de entrada e
  saída do aluno.
