\chapter{Introdução}

A disciplina de Requisitos de Software tem o objetivo de capacitar os alunos
quanto a compreensão de conceitos, técnicas, procedimentos e ferramentas que
ajudam a estabelecer requisitos necessários para a construção de um software.
As tarefas e técnicas que levam a um entendimento dos requisitos é denominado
engenharia de requisitos.

A função da engenharia de requisitos é compreender o conjunto de necessidades do
cliente a serem atendidas para solucionar um determinado problema do negócio em
que o cliente está inserido. No processo de software a engenharia de requisitos
inicia durante a atividade de comunicação e continua na de modelagem
\cite{pressman2016engenharia}, ou seja, faz uma ponte entre o projeto e a
construção. A satisfação do cliente quanto a entrega do software depende do
correto entendimento das necessidades e dos requisitos especificados.

\cite{sommerville2011engenharia} descreve requisitos como as descrições do que o
sistema deve fazer, os serviços que oferece e as restrições em seu
funcionamento. Sendo assim, todo projeto de software está sujeito às mudanças
que influenciam diretamente os requisitos, seja por um problema ou necessidade.
Pode-se dizer então que requisitos fazem parte de quase todo o ciclo de vida de
um software.

A partir de tal conhecimento, serão aplicado os conceitos de engenharia de
requisitos em um contexto real, desde o mapeamento do processo que será seguido
para a elicitação de requisitos até o desenvolvimento do software.

\section{Visão Geral do Relatório}

O presente relatório contará com a apresentação do contexto do negócio no qual
os requisitos serão identificados (Tópico 2), com a identificação e a descrição
do problema, das necessidades, das características, das estórias de usuário, dos
épicos, dos requisitos funcionais e dos requisitos não funcionais segundo a
abordagem de ER escolhida (Tópico 3), com o registro da experiência na execução
das técnicas de elicitação de requisitos definidas anteriormente na primeira
parte do trabalho (Tópico 4), com o detalhamento das estórias de usuário (Tópico
5), com o registro dos requisitos e de sua rastreabilidade na ferramenta
selecionada (Tópico 6), com o relato da experiência da execução do trabalho
(Tópico 7) e com o relato da experiência na disciplina de ER (Tópico 8).
